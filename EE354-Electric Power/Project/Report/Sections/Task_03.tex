\section{Task 3: Voltage Sensitivity Analysis}
\label{sec:task3}

\subsection{Method}
Each load bus (\(5,6,8\)) was changed independently by scaling both active and reactive demand with:
\[
\alpha \in \{-10\%,0\%,+10\%\},
\]
while all other loads were kept fixed. For each case, the full NR solver was run and all bus-voltage magnitudes were recorded.

For each target load bus, voltage spread was quantified using:
\begin{align}
\mathrm{Var}(V_b) &= \frac{1}{N}\sum_{m=1}^{N}\left(V_{b,m}-\bar V_b\right)^2, \\
\sigma(V_b) &= \sqrt{\mathrm{Var}(V_b)}.
\end{align}
To rank overall influence:
\[
\mathrm{RMS\_{STD}}_k = \sqrt{\frac{1}{n_b}\sum_{b=1}^{n_b}\sigma_b^2}.
\]

\subsection{Voltage Results Under Each Load Variation}
\begin{table}[H]
  \centering
  \caption{Voltage magnitudes for target load bus 5 variation.}
  \label{tab:task3_bus5}
  \scriptsize
  \begin{tabular}{crrrrrrrrrr}
    \toprule
    \(\Delta P,\Delta Q\) & \(V_1\) & \(V_2\) & \(V_3\) & \(V_4\) & \(V_5\) & \(V_6\) & \(V_7\) & \(V_8\) & \(V_9\) \\
    \midrule
    -10\% & 1.040000 & 1.025000 & 1.025000 & 1.027784 & 1.001712 & 1.014125 & 1.027257 & 1.017024 & 1.032976 \\
    0\%   & 1.040000 & 1.025000 & 1.025000 & 1.025788 & 0.995631 & 1.012654 & 1.025769 & 1.015883 & 1.032353 \\
    +10\% & 1.040000 & 1.025000 & 1.025000 & 1.023640 & 0.989300 & 1.011062 & 1.024214 & 1.014681 & 1.031688 \\
    \bottomrule
  \end{tabular}
  \normalsize
\end{table}

\begin{table}[H]
  \centering
  \caption{Voltage magnitudes for target load bus 6 variation.}
  \label{tab:task3_bus6}
  \scriptsize
  \begin{tabular}{crrrrrrrrrr}
    \toprule
    \(\Delta P,\Delta Q\) & \(V_1\) & \(V_2\) & \(V_3\) & \(V_4\) & \(V_5\) & \(V_6\) & \(V_7\) & \(V_8\) & \(V_9\) \\
    \midrule
    -10\% & 1.040000 & 1.025000 & 1.025000 & 1.026885 & 0.996397 & 1.016802 & 1.026146 & 1.016519 & 1.033273 \\
    0\%   & 1.040000 & 1.025000 & 1.025000 & 1.025788 & 0.995631 & 1.012654 & 1.025769 & 1.015883 & 1.032353 \\
    +10\% & 1.040000 & 1.025000 & 1.025000 & 1.024618 & 0.994804 & 1.008384 & 1.025370 & 1.015219 & 1.031402 \\
    \bottomrule
  \end{tabular}
  \normalsize
\end{table}

\begin{table}[H]
  \centering
  \caption{Voltage magnitudes for target load bus 8 variation.}
  \label{tab:task3_bus8}
  \scriptsize
  \begin{tabular}{crrrrrrrrrr}
    \toprule
    \(\Delta P,\Delta Q\) & \(V_1\) & \(V_2\) & \(V_3\) & \(V_4\) & \(V_5\) & \(V_6\) & \(V_7\) & \(V_8\) & \(V_9\) \\
    \midrule
    -10\% & 1.040000 & 1.025000 & 1.025000 & 1.025382 & 0.995196 & 1.012385 & 1.026947 & 1.019141 & 1.033334 \\
    0\%   & 1.040000 & 1.025000 & 1.025000 & 1.025788 & 0.995631 & 1.012654 & 1.025769 & 1.015883 & 1.032353 \\
    +10\% & 1.040000 & 1.025000 & 1.025000 & 1.026100 & 0.995931 & 1.012799 & 1.024507 & 1.012510 & 1.031298 \\
    \bottomrule
  \end{tabular}
  \normalsize
\end{table}

\subsection{Variance and Standard Deviation Tables}
\begin{table}[H]
  \centering
  \caption{Variance and standard deviation across buses for target load bus 5.}
  \label{tab:task3_varstd5}
  \begin{tabular}{ccc}
    \toprule
    Observed Bus & Variance & Std Dev \\
    \midrule
    1 & 0.0000000000 & 0.0000000000 \\
    2 & 0.0000000000 & 0.0000000000 \\
    3 & 0.0000000000 & 0.0000000000 \\
    4 & 0.0000028635 & 0.0016921912 \\
    5 & 0.0000256833 & 0.0050678664 \\
    6 & 0.0000015649 & 0.0012509498 \\
    7 & 0.0000015438 & 0.0012424893 \\
    8 & 0.0000009147 & 0.0009564155 \\
    9 & 0.0000002765 & 0.0005257919 \\
    \bottomrule
  \end{tabular}
\end{table}

\begin{table}[H]
  \centering
  \caption{Variance and standard deviation across buses for target load bus 6.}
  \label{tab:task3_varstd6}
  \begin{tabular}{ccc}
    \toprule
    Observed Bus & Variance & Std Dev \\
    \midrule
    1 & 0.0000000000 & 0.0000000000 \\
    2 & 0.0000000000 & 0.0000000000 \\
    3 & 0.0000000000 & 0.0000000000 \\
    4 & 0.0000008571 & 0.0009257769 \\
    5 & 0.0000004233 & 0.0006506190 \\
    6 & 0.0000118106 & 0.0034366545 \\
    7 & 0.0000001004 & 0.0003169338 \\
    8 & 0.0000002819 & 0.0005308973 \\
    9 & 0.0000005836 & 0.0007639111 \\
    \bottomrule
  \end{tabular}
\end{table}

\begin{table}[H]
  \centering
  \caption{Variance and standard deviation across buses for target load bus 8.}
  \label{tab:task3_varstd8}
  \begin{tabular}{ccc}
    \toprule
    Observed Bus & Variance & Std Dev \\
    \midrule
    1 & 0.0000000000 & 0.0000000000 \\
    2 & 0.0000000000 & 0.0000000000 \\
    3 & 0.0000000000 & 0.0000000000 \\
    4 & 0.0000000863 & 0.0002937266 \\
    5 & 0.0000000911 & 0.0003017733 \\
    6 & 0.0000000295 & 0.0001718018 \\
    7 & 0.0000009928 & 0.0009963793 \\
    8 & 0.0000073304 & 0.0027074721 \\
    9 & 0.0000006908 & 0.0008311667 \\
    \bottomrule
  \end{tabular}
\end{table}

\begin{table}[H]
  \centering
  \caption{Sensitivity ranking using aggregate voltage-spread indices.}
  \label{tab:task3_ranking}
  \begin{tabular}{cccc}
    \toprule
    Target Load Bus & Mean Std Dev & Max Std Dev & RMS Std Dev \\
    \midrule
    5 & 0.001193 & 0.005068 & 0.001910 \\
    6 & 0.000736 & 0.003437 & 0.001250 \\
    8 & 0.000589 & 0.002707 & 0.001012 \\
    \bottomrule
  \end{tabular}
\end{table}

\subsection{Task 3 Figures}
\begin{figure}[H]
  \centering
  \includegraphics[width=\linewidth]{\detokenize{Figures/task3_voltage_profiles.png}}
  \caption{Voltage profile variation across buses for each target load perturbation.}
  \label{fig:task3_profiles}
\end{figure}

\begin{figure}[H]
  \centering
  \includegraphics[width=0.8\linewidth]{\detokenize{Figures/task3_sensitivity_ranking.png}}
  \caption{Sensitivity ranking based on RMS standard deviation.}
  \label{fig:task3_rank}
\end{figure}

\subsection{Discussion of Findings}
The sensitivity ranking in Table~\ref{tab:task3_ranking} shows a clear order: bus 5 has the highest network-wide voltage influence, followed by bus 6, then bus 8. This ordering is physically reasonable for the given IEEE 9-bus topology and loading pattern. Bus 5 is connected to bus 4 and bus 7 paths that carry significant active and reactive transfers, so changing this load causes larger redistribution of currents and voltage drops in multiple corridors.

The variation tables indicate that generator buses 1--3 remain fixed at their voltage setpoints for all perturbations, which matches PV/slack bus modeling. Most sensitivity appears on PQ buses, especially near the disturbed load bus and along electrically close paths. For the bus-5 perturbation case, \(V_5\) has the largest spread (\(\sigma=0.005068\)), while nearby buses 4, 6, and 7 also show notable changes.

For the bus-6 load variation case, the largest spread occurs at \(V_6\), and the propagated effect on buses 4, 5, 8, and 9 is moderate. For the bus-8 case, \(V_8\) dominates the spread, but the global RMS impact is still smaller than the bus-5 case. This confirms that sensitivity is not only a function of local load size; it also depends on electrical coupling and reactive support paths.

From a planning viewpoint, these results suggest that bus 5 is the most critical candidate for voltage-support actions under demand uncertainty (e.g., local reactive compensation, dynamic VAR support, or tighter operational monitoring)~\cite{saadat1999power}. In contrast, bus-8 variations are relatively better contained and less disruptive to the whole system.

The trend is aligned with common load-flow interpretation in standard references: voltage sensitivity rises with stronger network coupling and weaker local reactive margin \cite{saadat1999power,bergen2000power,chapman2011electric}. Therefore, for operational studies on this system, load uncertainty at bus 5 should receive priority in contingency screening and voltage-security assessment.

Overall, Task 3 confirms that the developed NR solver can be reused for systematic parametric studies, not only for a single base-case solution. The same computational core used in Task 1 produced stable convergence for all 9 sensitivity cases, which supports the consistency and robustness of the implementation.

